\documentclass[12pt,fleqn]{article}
\usepackage{amsmath,amsthm,amssymb}
\usepackage[]{graphicx}
\usepackage[]{color}
\usepackage{listings}
\usepackage{fullpage}
\usepackage[]{hyperref}

%%%%%%%%%%%%%%%%%%%%%%%%%%%%%%%%%%%%%%%%%%%%%%%%%%%%%%%%%%%%
\DeclareMathOperator{\real}{Re}
\DeclareMathOperator{\imag}{Im}
\renewcommand{\i}{\mathrm{i}}
%%%%%%%%%%%%%%%%%%%%%%%%%%%%%%%%%%%%%%%%%%%%%%%%%%%%%%%%%%%%
\definecolor{gray}{rgb}{0.5,0.5,0.5}
\definecolor{dkgreen}{rgb}{.068,.578,.068}
\definecolor{dkpurple}{rgb}{.320,.064,.680}

% Matlab style, stolen from Chebfun coding style guide.
\lstset{
  language=Matlab,
  keywords={break,case,catch,continue,else,elseif,end,for,function,
    global,if,otherwise,persistent,return,switch,try,while},
  basicstyle=\footnotesize\ttfamily,
  keywordstyle=\color{blue}\bfseries,
  commentstyle=\color{dkgreen},
  stringstyle=\color{dkpurple},
  backgroundcolor=\color{white},
  frame=tlbr
  tabsize=4,
  showspaces=false,
  showstringspaces=false
}
%%%%%%%%%%%%%%%%%%%%%%%%%%%%%%%%%%%%%%%%%%%%%%%%%%%%%%%%%%%%

\begin{document}

\title{Potential theory with the Potential Theory Toolkit}
\author{E.~Kropf}
\date{\today}
\maketitle

\section{Introduction}
This document presents some complex potential theory using the Potential Theory Toolkit (PoTk) for MATLAB.

\subsection{Utility code}
The following definitions will be used in the subsequent.
\begin{lstlisting}
aspectequal = @(ax) set(ax, 'dataaspectratio', [1, 1, 1]);
streamColor = [0, 0.447, 0.741];
vectorColor = [0.929, 0.694, 0.125];
\end{lstlisting}

\section{Basic potential flow}
\subsection{Uniform flow}
The vector components of uniform flow at speed $U$ in the plane at an angle $\chi$ with respect ot he real axis are given by
\[ u = U\cos\chi \quad\text{and}\quad v = U\sin \chi, \]
from which we get
\begin{equation*}
  \frac{dW_U}{dz} = u - \i v = Ue^{-\i\chi}
\end{equation*}
so that the complex potential for the unform flow, up to a constant, is
\begin{equation*}
  W_U(z) = Uze^{-\i\chi}.
\end{equation*}
Computing this in a rectangle is given by
\begin{lstlisting}
U = 1;
chi = pi/4;

xylim = [-1, 1, -1, 1];
% Grid for streamlines.
z = rectgridz(xylim, 200);
% Grid for vector arrows.
zv = rectgridz(xylim, 20, 0.01);

Wz = U*exp(-1i*chi)*z;
vzv = 1i*U*exp(-1i*chi)*ones(size(zv));

quiver(real(zv), imag(zv), real(vzv), imag(vzv), 'color', vectorColor)
hold on
contour(real(z), imag(z), imag(Wz), 20, 'lineColor', streamColor)
\end{lstlisting}
and the result is shown in Figure~\ref{fig:simpleuniform}.
\begin{figure}[htb]
  \centering
  \includegraphics[height=.4\textheight]{figures/simpleuniform.eps}
  \caption{Uniform background flow at angle $\pi/4$.}
  \label{fig:simpleuniform}
\end{figure}

\subsubsection{Flow past one object}
Uniform flow past a circle is given by the Milne-Thompson circle theorem via
\begin{equation*}
  W(z) = U\left( z + \frac{1}{z} \right).
\end{equation*}
We may easily compute this via
\begin{lstlisting}
U = 1;
Wu = @(z, U) U*(z + 1./z);
dWdz = centdiffz(@(z) Wu(z, U));

xylim = 2.5*xylim;
z = rectgridz(xylim, 200);
outer = abs(z) > 1;
zv = rectgridz(xylim, 20, 0.01);
outv = abs(zv) > 1;

Wz = complex(nan(size(z)));
Wz(outer) = Wu(z(outer), U);
vzv = complex(nan(size(zv)));
vzv(outv) = dWdz(zv(outv));

quiver(real(zv), imag(zv), real(vzv), imag(vzv), 'color', vectorColor)
hold on
contour(real(z), imag(z), imag(Wz), 20, 'lineColor', streamColor)
\end{lstlisting}
\begin{figure}[htb]
  \centering
  \includegraphics[height=.4\textheight]{figures/aroundone}
  \caption{Simple flow around one obstacle.}
  \label{fig:aroundone}
\end{figure}

This flow may be modified by a line vortex via
\begin{equation*}
  W(z) = U\left( z + \frac{1}{z} \right) + \frac{\Gamma}{2\pi\i}\log z
\end{equation*}
since the effect is to maintain the boundary of the unit circle as a streamline.
Computing this we see
\begin{lstlisting}
Gamma = -3;
Wc = @(z, Gamma) Gamma/(2i*pi)*log(z);
W = @(z, U, Gamma) Wu(z, U) + Wc(z, Gamma);
dWdz = centdiffz(@(z) W(z, U, Gamma));

Wz(outer) = W(z(outer), U, Gamma);
vzv(outv) = dWdz(zv(outv));

quiver(real(zv), imag(zv), real(vzv), imag(vzv), 'color', vectorColor)
hold on
contour(real(z), imag(z), imag(Wz), 20, 'lineColor', streamColor)
fill(circle(0, 1))
plot(circle(0, 1))
\end{lstlisting}
Is it possible to compute the stagnation points?

We can force, by a specific value of $\Gamma$, to have the stagnation point at the bottom of the unit disk, $e^{\i 3\pi/2}$:
\begin{lstlisting}
Gamma = -4*pi;
dWdz = centdiffz(@(z) W(z, U, Gamma));

Wz(outer) = W(z(outer), U, Gamma);
vzv(outv) = dWdz(zv(outv));
\end{lstlisting}

If we further increase the circulation $\Gamma$, then the stagnation point moves off the circle:
\begin{lstlisting}
Gamma = -5*pi;
dWdz = centdiffz(@(z) W(z, U, Gamma));

xylim = [-2.5, 2.5, -3.5, 1.5];
z = rectgridz(xylim, 200);
outer = abs(z) > 1;
zv = rectgridz(xylim, 20, 0.01);
outvec = abs(zv) > 1;

Wz = complex(nan(size(z)));
Wz(outer) = W(z(outer), U, Gamma);
vzv = complex(nan(size(zv)));
vzv(outvec) = dWdz(zv(outvec));
\end{lstlisting}

\begin{figure}[htbp]
  \centering
  \includegraphics[width=.46\textwidth]{figures/onecirculate}
  \hspace{.02\textwidth}
  \includegraphics[width=.46\textwidth]{figures/circlebottom}
  \\\vspace*{.02\textwidth}
  \includegraphics[width=.46\textwidth]{figures/offcircle}
  \caption{Uniform flow past a cylinder with various circulations. Clockwise, from top-left, $\Gamma=-3$, $\Gamma=-4\pi$, and $\Gamma=-5\pi$.}
  \label{fig:onecirculate}
\end{figure}

%%fakesection Appendix
\clearpage
\appendix

\section{Utility functions}
\paragraph{Rectangular grid}
This function returns points in a rectangular grid.
\begin{lstlisting}
function z = rectgridz(xylim, res, pad)
%rectgridz contructs rectangular complex grid.

if nargin < 2
    res = 200;
end
if numel(res) == 1
    resx = res;
    resy = res;
else
    resx = res(1);
    resy = res(2);
end

if nargin < 3
    pad = 0;
end
if pad ~= 0
    padx = pad*diff(xylim(1:2));
    pady = pad*diff(xylim(3:4));
    xylim = [xylim(1) + padx, xylim(2) - padx, ...
        xylim(3) + pady, xylim(4) - pady];
end

[x, y] = meshgrid(linspace(xylim(1), xylim(2), resx), ...
    linspace(xylim(3), xylim(4), resy));
z = complex(x, y);
\end{lstlisting}

\paragraph{Centered difference}
A very simple centered difference function.
\begin{lstlisting}
function dw = centdiffz(f, h)
%centdiffz is the central difference formula for f(z).

if nargin < 2
    h = 1e-8;
end

dh = 0.5*h;
dw = @(z) (imag(f(z + 1i*dh) - f(z - 1i*dh)) ...
    - 1i*imag(f(z + dh) - f(z - dh)))/h;
\end{lstlisting}

\end{document}
