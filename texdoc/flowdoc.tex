% Everett Kropf, 2015
% 
% This file is part of the Potential Toolkit (PoTk).
% 
% PoTk is free software: you can redistribute it and/or modify
% it under the terms of the GNU General Public License as published by
% the Free Software Foundation, either version 3 of the License, or
% (at your option) any later version.
% 
% PoTk is distributed in the hope that it will be useful,
% but WITHOUT ANY WARRANTY; without even the implied warranty of
% MERCHANTABILITY or FITNESS FOR A PARTICULAR PURPOSE.  See the
% GNU General Public License for more details.
% 
% You should have received a copy of the GNU General Public License
% along with PoTk.  If not, see <http://www.gnu.org/licenses/>.

\documentclass[12pt,fleqn]{article}

\usepackage{amsmath,amsthm,amssymb}
\usepackage[]{graphicx}
\usepackage[]{color}
\usepackage{listings}
\usepackage{fullpage}
\usepackage[]{hyperref}

%%%%%%%%%%%%%%%%%%%%%%%%%%%%%%%%%%%%%%%%%%%%%%%%%%%%%%%%%%%%
\DeclareMathOperator{\real}{Re}
\DeclareMathOperator{\imag}{Im}
\newcommand{\conj}[1]{\overline{#1}}
\newcommand{\uconj}[1]{1/\conj{#1}}
\renewcommand{\i}{\mathrm{i}}
%%%%%%%%%%%%%%%%%%%%%%%%%%%%%%%%%%%%%%%%%%%%%%%%%%%%%%%%%%%%
\definecolor{gray}{rgb}{.5,.5,.5}
\definecolor{dkgreen}{rgb}{.068,.578,.068}
\definecolor{dkpurple}{rgb}{.320,.064,.680}
\definecolor{magenta}{rgb}{.619,.109,.302}

\lstset{
  language=Matlab,
  keywords={break,case,catch,continue,else,elseif,end,for,function,
    global,if,otherwise,persistent,return,switch,try,while},
  basicstyle=\footnotesize\ttfamily,
  keywordstyle=\color{blue}\bfseries,
  commentstyle=\color{dkgreen},
  stringstyle=\color{dkpurple},
  backgroundcolor=\color{white},
  frame=single,
  frameround=tttt,
  morekeywords=[2]{circle},
  keywordstyle=[2]\color{magenta}\bfseries,
  xleftmargin=.02\textwidth,
  xrightmargin=.02\textwidth
}
%%%%%%%%%%%%%%%%%%%%%%%%%%%%%%%%%%%%%%%%%%%%%%%%%%%%%%%%%%%%

\begin{document}

\title{Potential theory with the Potential Theory Toolkit}
\author{E.~Kropf}
\date{\today}
\maketitle

\section{Introduction}
This document presents some complex potential theory using the Potential Theory Toolkit (PoTk) for MATLAB.

\subsection{Utility code}
The following definitions will be used in the subsequent.
\begin{lstlisting}
aspectequal = @(ax) set(ax, 'dataaspectratio', [1, 1, 1]);
streamColor = [0, 0.447, 0.741];
vectorColor = [0.929, 0.694, 0.125];
\end{lstlisting}
See also the set of ancillary functions given in Appendix~\ref{sec:utility}.

\section{Basic potential flow}
\subsection{Uniform flow}
The vector components of uniform flow at speed $U$ in the plane at an angle $\chi$ with respect to the real axis are given by
\[ u = U\cos\chi \quad\text{and}\quad v = U\sin \chi, \]
from which we get
\begin{equation*}
  \frac{dW_U}{dz} = u - \i v = Ue^{-\i\chi}
\end{equation*}
so that the complex potential for the unform flow, up to a constant, is
\begin{equation*}
  W_U(z) = Uze^{-\i\chi}.
\end{equation*}
Computing this in a rectangle is given by
\begin{lstlisting}
U = 1;
chi = pi/4;

xylim = [-1, 1, -1, 1];
% Grid for streamlines.
z = rectgridz(xylim, 200);
% Grid for vector arrows.
zv = rectgridz(xylim, 20, 0.01);

Wz = U*exp(-1i*chi)*z;
vzv = 1i*U*exp(-1i*chi)*ones(size(zv));

quiver(real(zv), imag(zv), real(vzv), imag(vzv), 'color', vectorColor)
hold on
contour(real(z), imag(z), imag(Wz), 20, 'lineColor', streamColor)
\end{lstlisting}
and the result is shown in Figure~\ref{fig:simpleuniform}.
\begin{figure}[htb]
  \centering
  \includegraphics[height=.4\textheight]{figures/simpleuniform.eps}
  \caption{Uniform background flow at angle $\pi/4$.}
  \label{fig:simpleuniform}
\end{figure}

\subsubsection{Flow past one object}
Uniform flow past a circle is given by the Milne-Thompson circle theorem via
\begin{equation*}
  W(z) = U\left( z + \frac{1}{z} \right).
\end{equation*}
We may easily compute this via
\begin{lstlisting}
U = 1;
Wu = @(z, U) U*(z + 1./z);
dWdz = centdiffz(@(z) Wu(z, U));

xylim = 2.5*xylim;
z = rectgridz(xylim, 200);
outer = abs(z) > 1;
zv = rectgridz(xylim, 20, 0.01);
outv = abs(zv) > 1;

Wz = complex(nan(size(z)));
Wz(outer) = Wu(z(outer), U);
vzv = complex(nan(size(zv)));
vzv(outv) = dWdz(zv(outv));

quiver(real(zv), imag(zv), real(vzv), imag(vzv), 'color', vectorColor)
hold on
contour(real(z), imag(z), imag(Wz), 20, 'lineColor', streamColor)
fill(circle(0, 1))
plot(circle(0, 1))
\end{lstlisting}
\begin{figure}[htb]
  \centering
  \includegraphics[height=.4\textheight]{figures/aroundone}
  \caption{Simple flow around one obstacle.}
  \label{fig:aroundone}
\end{figure}

This flow may be modified by a line vortex via
\begin{equation*}
  W(z) = U\left( z + \frac{1}{z} \right) + \frac{\Gamma}{2\pi\i}\log z
\end{equation*}
since the effect is to maintain the boundary of the unit circle as a streamline.
Computing this we see
\begin{lstlisting}
Gamma = -3;
Wc = @(z, Gamma) Gamma/(2i*pi)*log(z);
W = @(z, U, Gamma) Wu(z, U) + Wc(z, Gamma);
dWdz = centdiffz(@(z) W(z, U, Gamma));

Wz(outer) = W(z(outer), U, Gamma);
vzv(outv) = dWdz(zv(outv));

quiver(real(zv), imag(zv), real(vzv), imag(vzv), 'color', vectorColor)
hold on
contour(real(z), imag(z), imag(Wz), 20, 'lineColor', streamColor)
fill(circle(0, 1))
plot(circle(0, 1))
\end{lstlisting}
Is it possible to compute the stagnation points?

We can force, by a specific value of $\Gamma$, to have the stagnation point at the bottom of the unit disk, $e^{\i 3\pi/2}$ (from here forward we will not print the plotting commands, unless something new is added):
\begin{lstlisting}
Gamma = -4*pi;
dWdz = centdiffz(@(z) W(z, U, Gamma));

Wz(outer) = W(z(outer), U, Gamma);
vzv(outv) = dWdz(zv(outv));
\end{lstlisting}

If we further increase the circulation $\Gamma$, then the stagnation point moves off the circle:
\begin{lstlisting}
Gamma = -5*pi;
dWdz = centdiffz(@(z) W(z, U, Gamma));

xylim = [-2.5, 2.5, -3.5, 1.5];
z = rectgridz(xylim, 200);
outer = abs(z) > 1;
zv = rectgridz(xylim, 20, 0.01);
outvec = abs(zv) > 1;

Wz = complex(nan(size(z)));
Wz(outer) = W(z(outer), U, Gamma);
vzv = complex(nan(size(zv)));
vzv(outvec) = dWdz(zv(outvec));
\end{lstlisting}

\begin{figure}[htbp]
  \centering
  \includegraphics[width=.46\textwidth]{figures/onecirculate}
  \hspace{.02\textwidth}
  \includegraphics[width=.46\textwidth]{figures/circlebottom}
  \\\vspace*{.02\textwidth}
  \includegraphics[width=.46\textwidth]{figures/offcircle}
  \caption{Uniform flow past a cylinder with various circulations. Clockwise, from top-left, $\Gamma=-3$, $\Gamma=-4\pi$, and $\Gamma=-5\pi$.}
  \label{fig:onecirculate}
\end{figure}

\clearpage

\section{Multiply connected domains}
Let $D_\zeta$ be the unit disk with $m$ disconnected circular holes. We let the pairs $\{\delta_j,q_j\}_{j=1}^m$ represent the centres and radii of these holes. For indices $j=\{0,1,\dots,m\}$ we denote the $j$th boundary circle of $\partial D_\zeta$ by $C_j$, where $C_0$ is the unit circle.
Let $C_j'$ be the reflection of circle $C_j$ through the unit circle. It can be shown the M\"obius transformation
\begin{equation*}
  \theta_j(\zeta) = \delta_j + \frac{q_j^2 \zeta}{1 - \conj{\delta_j}\zeta}
\end{equation*}
maps points on the outer circle $C_j'$ to points on the inner circle $C_j$.
Let this sentence stand in as the definition of the first kind integrals $v_j(\zeta)$. Let this sentence stand in as the introduction to the Schottky-Klein prime function.
For more on this and information on computing the Schottky-Klein prime function $\omega$, see \cite{primecalc}.

For flow in multiply connected domains, there are actually two cases to consider. The bounded domain and the unbounded domain. We first consider the former.

\subsection{Bounded flow region}
From \cite{greensfunctions} we note the $m+1$ functions, for $\alpha$ and $\zeta$ in $D_\zeta$,
\begin{equation}
  G_j(\zeta,\alpha) = \frac{1}{2\pi\i} \log \left[ \frac{q_j}{|\alpha - \delta_j|} \frac{\omega(\zeta,\alpha)}{\omega(\zeta,\theta_j(\uconj{\alpha}))} \right]
  \label{eq:greensfunction}
\end{equation}
have the following properties: The function $G_j$ is the complex potentential such that
\begin{enumerate}
  \item there exists a point vortex of strength 1 at $\alpha$, and
  \item the boundary $C_j$ has a circulation of strength -1.
\end{enumerate}
In the special case when $j=0$ and $\alpha=0$ we note the Green's function takes the form
\begin{equation*}
  G_0(\zeta,0) = \frac{1}{2\pi\i} \log\left[ \frac{\omega(\zeta,0)}{\omega(\zeta,\infty)} \right].
\end{equation*}
It will be beneficial in the subsequent to note we never need to compute $G_j$ for $j>1$ directly, since by a Lemma in \cite{primecalc} we have
\begin{equation}
  G_j(\zeta,\alpha) = G_0(\zeta,\alpha) - v_j(\zeta) + K_j(\alpha)
  \label{eq:greens_identity}
\end{equation}
where
\begin{equation*}
  K_j(\alpha) = \conj{v_j(\alpha)} + \tfrac{1}{2}\tau_{jj} + \frac{1}{2\pi}\arg\left[ \frac{\alpha}{\alpha - \delta_j} \right]
\end{equation*}
is a constant dependent solely on $\alpha$. Since complex potential is determined up to a constant, we ignore it in computation.

The flow due to $n$ point vortices at locations $\{\alpha_k:1\le k\le n\}$ with strengths denoted by scalars $\Gamma_k$ may, based on the properties discussed above, simply be given by a sum of the terms $\Gamma_k G_0(\zeta,\alpha_k)$. This immediately places a circulation of $-\sum_{k=1}^n \Gamma_k$ on boundary $C_0$, but we will ignore this momentarily. To apply a circulation to the boundaries $\{C_j:0<j\le m\}$, we pick any point $\beta\in D_\zeta$ and use the linear combination $-\sum_{j=1}^m \gamma_j G_j(\zeta,\beta)$, where $\gamma_j$ is the strength of the circulation. This of course puts a possibly unwanted circulation of $-\sum \gamma_j$ at the point $\beta$, but we may move this unwanted flow to $C_0$ by an associated linear combination of the $G_0$ functions. In fact we have now placed all of our incidental circulation at the boundary $C_0$, but this is common in flow problems, and it is easy to move this flow to another boundary or point as desired by the appropriate combination of Green's functions. Our expression for flow due to circulation is then given by
\begin{equation}
  \begin{split}
    W_\Gamma(\zeta) &= \sum_{k=1}^n \Gamma_k G_0(\zeta,\alpha_k) + \sum_{j=1}^m \gamma_j \left[ G_0(\zeta,\beta) - G_j(\zeta,\beta) \right] \\
    &= \sum_{k=1}^n \Gamma_k G_0(\zeta,\alpha_k) + \sum_{j=1}^m \gamma_j v_j(\zeta).
  \end{split}
  \label{eq:bvorticesmc}
\end{equation}
Note the final form of the function does not involve the point $\beta$ or, by application of equation~\eqref{eq:greens_identity}, any of the functions $\{G_j:0<j\le m\}$.

A uniform field, or background flow, in the bounded domain is represented by a point dipole located at some $\beta\in D_\zeta$ acting as a source and sink.
Following \cite{newcalculus}, we may write down that this flow is given by
\begin{equation}
  W_U(\zeta) = -4\pi U\left. \left( \frac{\partial G_0(\zeta,\alpha)}{\partial \alpha_y}\cos\chi + \frac{\partial G_0(\zeta,\alpha)}{\partial \alpha_x}\sin\chi \right) \right|_{\alpha = \beta}
  \label{eq:backgroudmc}
\end{equation}
where we have used that $\alpha = \alpha_x + \i\alpha_y$.
Finally, the combined flow is written as above by simply $W(\zeta) = W_U(\zeta) + W_\Gamma(\zeta)$.

\subsection{Unbounded flow region}
Let $\Omega$ be an $m+1$ connected unbounded domain where all of the component boundaries are circles. Then there is a M\"obius transformation $\zeta(z)$ with the mapping $\Omega\mapsto D_\zeta$. We denote the $j$th circle of $\partial\Omega$ by $R_j$, and the map $\zeta(z)$ associates $R_j$ with $C_j$. Let the points $\{\alpha_k:1\le k\le n\}$ be images under $\zeta$ of the $n$ point vortices of strength $\Gamma_k$ located in $\Omega$, and we will allow for circulation around the boundaries of $\Omega$ with strengths given by $\{\gamma_j:0\le j\le m\}$. And finally, define the point $\beta = \zeta(\infty)$.

Unlike the bounded case, we want to be able to specify the flow around all of the boundaries, but we also have the freedom of the point at infinity, which we did not previously have. Proceding as in the bounded case, but now moving all of the unwanted circulation to the point at infinity, we write the potential due to circulation is given by
\begin{equation}
  \begin{split}
    W_\Gamma(\zeta) &= \sum_{k=1}^n \Gamma_k \left[ G_0(\zeta,\alpha_k) - G_0(\zeta,\beta) \right] - \sum_{j=0}^m \gamma_j G_j(\zeta,\beta) \\
    &= \sum_{k=1}^n \Gamma_k G_0(\zeta,\alpha_k) + \sum_{j=1}^m \gamma_j v_j(\zeta) - \left[ \sum_{k=1}^n \Gamma_k + \sum_{j=0}^m \gamma_j \right] G_0(\zeta,\beta).
  \end{split}
  \label{eq:vorticesmc}
\end{equation}
The relationship between equation~\eqref{eq:vorticesmc} and equation~\eqref{eq:bvorticesmc} is exploited in the architecture of PoTk, and the bounded domain flow is always computed as a starting point.

The background flow in the unbounded case is given by treating the point at infinity as a dipole. Since this is associated with the point $\beta\in D_\zeta$, the same expression for $W_U$ is used here as in the bounded case. The combined flow is again $W_U + W_\Gamma$.

%%fakesection References
\bibliographystyle{abbrv}
\bibliography{flowref}

%%fakesection Appendix
\clearpage
\appendix

\section{Utility functions}\label{sec:utility}
\paragraph{Rectangular grid}
This function returns points in a rectangular grid.
\begin{lstlisting}
function z = rectgridz(xylim, res, pad)
%rectgridz contructs rectangular complex grid.

if nargin < 2
    res = 200;
end
if numel(res) == 1
    resx = res;
    resy = res;
else
    resx = res(1);
    resy = res(2);
end

if nargin < 3
    pad = 0;
end
if pad ~= 0
    padx = pad*diff(xylim(1:2));
    pady = pad*diff(xylim(3:4));
    xylim = [xylim(1) + padx, xylim(2) - padx, ...
        xylim(3) + pady, xylim(4) - pady];
end

[x, y] = meshgrid(linspace(xylim(1), xylim(2), resx), ...
    linspace(xylim(3), xylim(4), resy));
z = complex(x, y);
\end{lstlisting}

\paragraph{Centered difference}
A very simple centered difference function.
\begin{lstlisting}
function dw = centdiffz(f, h)
%centdiffz is the central difference formula for f(z).

if nargin < 2
    h = 1e-8;
end

dh = 0.5*h;
dw = @(z) (imag(f(z + 1i*dh) - f(z - 1i*dh)) ...
    - 1i*imag(f(z + dh) - f(z - dh)))/h;
\end{lstlisting}

\end{document}
